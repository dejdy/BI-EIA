\documentclass[a4paper,11pt]{article}
\usepackage[utf8]{inputenc}
\usepackage[T1]{fontenc}
\usepackage[czech]{babel}
\usepackage{hyperref}
\usepackage{listings}
\usepackage{mathtools}
\usepackage{graphicx}

\title{In-place Mergesort}
\author{Dejdar Jan}
\date{8.~11. 2016}


%%%%%%%%%%%%%%%%%%%%%%%%%%%%%%%%%%%%%%%%%%%%%%%%%%%%%%%%%
%%%%%%%%%%%%%%%%%%%%%%%%%%%%%%%%%%%%%%%%%%%%%%%%%%%%%%%%%

\begin{document}

\maketitle

%%%%%%%%%%%%%%%%%%%%%%%%%%%%%%%%%%%%%%%%%%%%%%%%%%%%%%%%%
%%%%%%%%%%%%%%%%%%%%%%%%%%%%%%%%%%%%%%%%%%%%%%%%%%%%%%%%%

\begin{section}{Definice problému}
Jednou z nevýhod populárního třídícího algoritmu merge sort je paměťová
složitost $O(n\log{n})$ u neoptimalizované verze, případně $O(n)$ u implementace,
která si pracovní pole alokuje dopředu. Existuje však i in-place verze tohoto algoritmu,
kterou se zde budu zabývat. 

Efektivitu algoritmu se pokusím zvýšít nahrazením in-place verze algoritmu algoritmu 
merge sortem a insertion sort pro malá pole. Dále budu zkoumat vliv rozložení vstupní 
posloupnosti. Změřím časy jednotlivých verzí algoritmu pro náhodnou, seřazenou a obráceně
seřazenou vstupní posloupnost.

\begin{subsection}{Popis sekvenčního algoritmu}
Klasický merge sort používá přídavné pomocné pole pro slévaní dvouseřazených částí pole.
V implementaci in-place verze budeme muset použít zbytek původního pole jako pracovní prostor
pro toto slévání. Prvky v tomto zbytku pole ale není možné přepsat, protože je budeme řadit později.
Myšlenka jak toho docílit spočívá v tom, že když chceme menší z dvou právě porovnávaných prvků
umístit do pracovního prostoru, vyměníme tento prvek s příslušným prvkem v pracovním prostoru.
Po skončení slévání tedy dvě původně seřazená podpole obsahují prvky, které předtím byly v
pracovním prostoru, zatímco tento obsahuje seřazené pole.

Při slévání musí být splněny dvě podmínky:
\begin{enumerate}
	\item Pracovní prostor musí být dostatečně velký, aby pojal obě části pole, které chceme slévat.
	\item Pracovní prostor se může překrývat se seřazenými podpoli, ale je nutné zajistit, že nepřepíšeme
		žádné neslité prvky.
\end{enumerate}
\pagebreak

S výše algoritmem je triviální sestrojit algoritmus, který seřadí polovinu původního pole, 
protože máme dostatek pracovního prostoru pro slévání seřazených prvků. Otázkou zůstává, jak
seřadit zbývající neseřazenou polovinu pole.


\begin{table}[ht]
\centering
\begin{tabular}{|c|c|}
... SEŘAZENÉ ... & ... NESEŘAZENÉ ...
\end{tabular}
\caption{Seřazená polovina pole}
\end{table}

Intuitivní postup by byl rekurzivně seřadit druhou polovinu neseřazeného pole, takže by zbyla jen $\frac{1}{4}$
neseřazeného pole. Problém je v tom, že musíme slít $B$ a $A$, na což nemáme dostatečně velký pracovní prostor.


\begin{table}[ht]
\centering
\begin{tabular}{|c|c|c|}
... NESEŘAZENÁ $\frac{1}{4}$ ... & ... SEŘAZENÁ $\frac{1}{4}$ ($B$) ... & ... SEŘAZENÁ $\frac{1}{2}$ ($A$) ...
\end{tabular}
\caption{Na slití $A$ a $B$ nemáme dostatečný pracovní prostor}
\end{table}

Klíčová myšlenka spočívá v tom, že místo abychom seřadili druhou polovinu neseřazeného pole, seřadíme tu první,
čímž umístíme pracovní prostor mezi dvě seřazené části pole. Pracovní prostor se pak bude překrývat se seřazenou 
polovinou pole $A$.

Nyní uvažujme dva extrémní případy, které mohou nastat:
\begin{enumerate}
	\item Všechny prvky v B jsou menší než prvky v A. Potom slívací algoritmus postupně přesune $B$ do pracovního
prostoru. Protože jejich velikosti jsou stejné, vše je v opřádku.

	\item Všechny prvky v B jsou větší než v A. V tomto případě slívací algoritmus postupně přesouvá prvky z $A$
do pracovního prostoru. Po zaplnění celého původního pracovního prostoru začne algoritmus přepisovat první polovinu
pole $A$. Nejedná se ale o neslité prvky, takže podle druhé podmínky je vše v pořádku. Pracovní pole se tak tedy 
přesune na pravou stranu. Nyní algoritmus začne prohazovat prvky z $B$ s pracovním prostorem. Pracovní pole se tak 
tedy přesune na levou stranu, viz tabulku.
\end{enumerate}

\begin{table}[ht]
\centering
\begin{tabular}{|c|c|c|}
... SEŘAZENÁ $\frac{1}{4}$ ($B$) ... & ... PRACOVNÍ PROSTOR $\frac{1}{4}$ ... & ... SEŘAZENÁ $\frac{1}{2}$ ($A$) ...
\end{tabular}
\begin{center}
$\Downarrow$
\end{center}
\begin{tabular}{|c|c|}
... PRACOVNÍ PROSTOR $\frac{1}{4}$ ($B$) ... & ... SLITÉ $\frac{3}{4}$ ...
\end{tabular}

\caption{Slití $A$ a $B$}
\end{table}

Takto algoritmus pokračuje a rekurzivně zmenšuje a postupně slévá seřazená pole. Když se velikost pole zmenší pod určitou
mez, můžeme přepnout na jiný řadící algoritmus (standardní merge sort, Insertion sort...) a tím se pokusit optimalizovat
rychlost výsledného algoritmu.



\end{subsection}
\end{section}

\end{document}
