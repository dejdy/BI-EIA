\documentclass[a4paper,11pt]{article}
\usepackage[utf8]{inputenc}
\usepackage[T1]{fontenc}
\usepackage[czech]{babel}
\usepackage{hyperref}
\usepackage{listings}
\usepackage{mathtools}
\usepackage{graphicx}

\title{In-place Mergesort}
\author{Dejdar Jan}
\date{8.~11. 2016}


%%%%%%%%%%%%%%%%%%%%%%%%%%%%%%%%%%%%%%%%%%%%%%%%%%%%%%%%%
%%%%%%%%%%%%%%%%%%%%%%%%%%%%%%%%%%%%%%%%%%%%%%%%%%%%%%%%%

\begin{document}

\maketitle

%%%%%%%%%%%%%%%%%%%%%%%%%%%%%%%%%%%%%%%%%%%%%%%%%%%%%%%%%
%%%%%%%%%%%%%%%%%%%%%%%%%%%%%%%%%%%%%%%%%%%%%%%%%%%%%%%%%

\begin{section}{Definice problému}
Jednou z nevýhod populárního třídícího algoritmu Merge sort je paměťová
složitost $O(n\log{n})$ u neoptimalizované verze, případně $O(n)$ u implementace,
která si pracovní pole alokuje dopředu. Existuje však i in-place verze tohoto algoritmu,
kterou se zde budu zabývat. 

Efektivitu algoritmu se pokusím zvýšit použitím algoritmu Insertion sort pro malá pole.
Pokusím se najít optimální mez pro spuštění Insertion sortu. Dále budu zkoumat vliv rozložení 
vstupní posloupnosti. Změřím časy jednotlivých verzí  algoritmu pro náhodnou a obráceně seřazenou 
vstupní posloupnost.

\begin{subsection}{Popis sekvenčního algoritmu}
Klasický Merge sort používá přídavné pomocné pole pro slévaní dvou seřazených částí pole.
V implementaci in-place verze budeme muset použít zbytek původního pole jako pracovní prostor
pro toto slévání. Prvky v tomto zbytku pole ale není možné přepsat, protože je budeme řadit později.
Myšlenka jak toho docílit spočívá v tom, že když chceme menší z dvou právě porovnávaných prvků
umístit do pracovního prostoru, vyměníme tento prvek s příslušným prvkem v pracovním prostoru.
Po skončení slévání tedy dvě původně seřazená podpole obsahují prvky, které předtím byly v
pracovním prostoru, zatímco tento obsahuje seřazené pole.

Při slévání musí být splněny dvě podmínky:
\begin{enumerate}
	\item Pracovní prostor musí být dostatečně velký, aby pojal obě části pole, které chceme slévat.
	\item Pracovní prostor se může překrývat se seřazenými podpoli, ale je nutné zajistit, že nepřepíšeme
		žádné neslité prvky.
\end{enumerate}
\pagebreak

S výše uvedeným postupem je triviální sestrojit algoritmus, který seřadí polovinu původního pole, 
protože máme dostatek pracovního prostoru pro slévání seřazených prvků. Výsledek je znázorněn v tabulce \ref{table:one}.
Otázkou zůstává, jak seřadit zbývající neseřazenou polovinu pole.


\begin{table}[ht]
\centering
\begin{tabular}{|c|c|}
... SEŘAZENÉ ... & ... NESEŘAZENÉ ...
\end{tabular}
\caption{Seřazená polovina pole}
\label{table:one}
\end{table}

Intuitivní postup by byl rekurzivně seřadit druhou polovinu neseřazeného pole (viz tabulka \ref{table:two}), takže by zbyla jen $\frac{1}{4}$
neseřazeného pole. Problém je v tom, že musíme slít $B$ a $A$, na což nemáme dostatečně velký pracovní prostor.


\begin{table}[ht]
\centering
\begin{tabular}{|c|c|c|}
... NESEŘAZENÁ $\frac{1}{4}$ ... & ... SEŘAZENÁ $\frac{1}{4}$ ($B$) ... & ... SEŘAZENÁ $\frac{1}{2}$ ($A$) ...
\end{tabular}
\caption{Na slití $A$ a $B$ nemáme dostatečný pracovní prostor}
\label{table:two}
\end{table}

Klíčová myšlenka spočívá v tom, že místo abychom seřadili druhou polovinu neseřazeného pole, seřadíme tu první,
čímž umístíme pracovní prostor mezi dvě seřazené části pole. Pracovní prostor se pak bude překrývat se seřazenou 
polovinou pole $A$.

Nyní uvažujme dva extrémní případy, které mohou nastat:
\begin{enumerate}
	\item Všechny prvky v $B$ jsou menší než prvky v $A$. Potom slívací algoritmus postupně přesune $B$ do pracovního
prostoru. Protože jejich velikosti jsou stejné, vše je v pořádku.

	\item Všechny prvky v $B$ jsou větší než v $A$. V tomto případě slívací algoritmus postupně přesouvá prvky z $A$
do pracovního prostoru. Po zaplnění celého původního pracovního prostoru začne algoritmus přepisovat první polovinu
pole $A$. Nejedná se ale o neslité prvky, takže podle druhé podmínky je vše v pořádku. Pracovní pole se tak tedy 
přesune na pravou stranu. Nyní algoritmus začne prohazovat prvky z $B$ s pracovním prostorem. Pracovní pole se tak 
tedy přesune na levou stranu (viz tabulka \ref{table:three}).
\end{enumerate}

\begin{table}[ht]
\centering
\begin{tabular}{|c|c|c|}
... SEŘAZENÁ $\frac{1}{4}$ ($B$) ... & ... PRACOVNÍ PROSTOR $\frac{1}{4}$ ... & ... SEŘAZENÁ $\frac{1}{2}$ ($A$) ...
\end{tabular}
\begin{center}
$\Downarrow$
\end{center}
\begin{tabular}{|c|c|}
... PRACOVNÍ PROSTOR $\frac{1}{4}$ ($B$) ... & ... SLITÉ $\frac{3}{4}$ ...
\end{tabular}

\caption{Slití $A$ a $B$}
\label{table:three}
\end{table}

Takto algoritmus pokračuje a rekurzivně zmenšuje a postupně slévá seřazená pole. Když se velikost pole zmenší pod určitou
mez, můžeme přepnout na jiný řadící algoritmus (například Insertion sort) a tím optimalizovat rychlost výsledného algoritmu.
\end{subsection}
\end{section}


\begin{section}{Optimalizovaná verze}
\begin{subsection}{Provedené optimalizace}
Algoritmus in-place merge sort je rekurzivní, tedy postupně dělí pole na menší a menší, která řadí a následně slévá dohromady.
Celková asymptotická složitost tohoto algoritmu je $O(n\log{n})$. Vzhledem k tomu, že asymptotická složitost nebere v úvahu
konstanty, může být pro malá $n$ ve skutečnosti výhodnější použít algoritmus s vyšší asymptotickou složitostí. Můžeme si 
například představit, že skutečná složitost algoritmu $A$ je $100~000*n\log{n}$ a algoritmu $B$ $0.1*n^2$. 
Asymptoticky je tedy lepší algoritmus $A$, avšak například pro \(n = 10\) bude zcela jistě výhodnější použít algoritmus $B$.

Z výše uvedeného důvodu jsem algoritmus in-place merge sort zoptimalizoval použitím algoritmu Insertion sort v určité
fázi rekurzivního dělení vstupního pole. Insertion sort má asymptotickou složitost $O(n^2)$, ale má oproti rekurzivnímu
merge sortu menší nároky na režii.

Důležitou otázkou bylo, jak zvolit hranici $h$ pro ukončení rekurzivního volání a spuštění algoritmu Insertion sort.
V tabulce \ref{table:four} jsou zaznamenány změřené časy pro různé hodnoty $h$. Měření probíhalo pro náhodnou, předem vygenerovanou
vygenerovanou posloupnost délky $100~000$. Všechny běhy programu měly tatáž vstupní data. Pro zajímavost je na konci
uveden naměřený čas pro klasický merge sort. Data potvrzují správnost úvahy výše, tedy že pro malá pole se v tomto 
případě vyplatí použít asymptoticky horší algoritmus Insertion sort.


\begin{table}[ht]\centering
\begin{tabular}{|c|c|c|}
\hline
$h$ & Algoritmus & Naměřený čas \texttt{[ms]}\\ \hline
\hline
1 & In-place Merge sort & 78,69\\
2 & Hybridní in-place Merge sort & 75,34\\
5 & Hybridní in-place Merge sort & 75,04\\
6 & Hybridní in-place Merge sort & 74,71\\
10 & Hybridní in-place Merge sort & 77,71\\
20 & Hybridní in-place Merge sort & 86,72\\
- & Klasický Merge sort & 28,19\\
\hline
\end{tabular}
\caption{Naměřené časy pro různá $h$, náhodné pole délky $100~000$}
\label{table:four}
\end{table}
\end{subsection}
\pagebreak

\begin{subsection}{Naměřené časy}
V tabulce níže jsou zaznamenány časy různých implementací.

\begin{table}[h]\centering
\begin{tabular}{|l|c|c|c|c|}
\hline
Algoritmus / Velikost vstupu & 100~000 & 500~000 & 1~000~000 & 10~000~000\\
\hline\hline
In-place Merge sort & 78,69~ms & 476,71~ms & 1,03~s & 13,716~s\\
Hybrid in-place Merge sort ($h$ = 6) & 74,71~ms & 456,55~ms & 0,979~s & 12,844~s\\
Classic Merge sort & 28,19~ms & 151,11~ms & 0,31~s & 3,382~s\\
\hline

\end{tabular}
\caption{Naměřené časy pro náhodnou posloupnost}
\label{table:five}
\end{table}


\begin{table}[h]\centering
\begin{tabular}{|l|c|c|c|c|}
\hline
Algoritmus / Velikost vstupu & 100~000 & 500~000 & 1~000~000 & 10~000~000\\
\hline\hline
In-place Merge sort & 56,43~ms & 343,49~ms & 751,58~ms & 10,382~s\\
Hybrid in-place Merge sort ($h$ = 6) & 45,4~ms & 284,96~ms & 628,215~ms & 8,222~s\\
Classic Merge sort & 14,89~ms & 82,91~ms & 173,015~ms & 1,997~s\\
\hline

\end{tabular}
\caption{Naměřené časy pro obráceně seřazenou posloupnost}
\label{table:six}
\end{table}
\end{subsection}

\begin{subsection}{Závěr}
Z naměřených časů je zřejmé, že použití algoritmu Insertion sort pro menší pole zrychlilo
celkový algoritmus. Celkové zrychlení je závislé na zvolení meze $h$ pro spuštění
algoritmu Insertion sort. Z měření nejlépe vyšla verze algoritmu pro $h$ = 6. Tato hodnota je tedy
uvedena v tabulkách a je použita pro další měření.

Pro náhodnou posloupnost je naměřené zrychlení kolem 6~\%, ale pro obráceně seřazenou posloupnost
se algoritmus zrychlil o více než 20~\%. Při velikosti vstupu $10^7$ se jedná o více než dvě vteřiny,
což je významný rozdíl.

Paměťová složitost $O(1)$ in-place verze algoritmu Merge sort je však vykoupena složitostí časovou.
Ve všech testech byl klasický Merge sort několikrát rychlejší než optimalizovaná in-place verze.
In-place verze provadí mnohem více prohození prvků, což se odráží právě na výsledné časové složitosti.
Z tohoto důvodu postrádá uplatnění v praxi, kde lineární paměťová složitost není takový problém.

\end{subsection}
\newpage
\begin{thebibliography}{}

\bibitem{bees}
Xinyu LIU, Larry \emph{Elementary Algorithms} [online].  Dostupné z: \url{http://bit.ly/2g39dc4}

\end{thebibliography}

\end{section}
\end{document}
